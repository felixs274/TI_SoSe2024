\documentclass[11pt, a4paper]{scrartcl}
\usepackage[utf8]{inputenc}
\usepackage[ngerman]{babel}
\usepackage[T1]{fontenc}
\usepackage{soul}
\usepackage[dvipsnames]{xcolor}
\usepackage{amsmath}
\usepackage{amssymb}
\usepackage{enumitem}
\usepackage{graphicx}
\graphicspath{{images/}}
\usepackage{tikz}
\usetikzlibrary{automata, positioning}


\title{Theoretische Informatik}
\author{Zusammenfassung}
\date{SoSe2024}

\begin{document}
\maketitle

\tableofcontents
\newpage


% --- START DES EIGENTLICHEN INHALTS -------------------------------------------------

\section{Allgemein}

\vspace{0.5em}

\subsection{Alphabete und W"orter}

\begin{itemize}
    \item Ein Alphabet $\Sigma$ ist eine endliche Menge unterscheidbarer Symbole
    \item Element $\sigma \in \Sigma$ ist ein Zeichen des Alphabets $\Sigma$
    \item Jedes Element $\omega \in \Sigma^*$  ist ein Wort "uber $\Sigma$
    \item $\varepsilon$ = Leeres Wort
    \item $\Sigma^*$: Menge aller W"orter "uber $\Sigma$
    \item $\Sigma^+$: Menge aller W"orter "uber $\Sigma$ mit mind. 1 Element
    \item $|\omega|$: L"ange eines Wortes ($|\varepsilon|$ = 0)
\end{itemize}

\vspace{1em}

\subsection{Grammatiken}

Eine Grammatik G ist ein 4-Tupel (V, $\Sigma$, P, S):
\begin{itemize}
    \item V: endliche Menge an Nicht-Terminal-Symbolen
    \item $\Sigma$: endliche Menge an Terminal-Symbolen (V$\cap\Sigma=\varnothing$)
    \item P: endliche Menge an Produktionsregeln
    \item S: Startsymbol (S$\in$V)
\end{itemize}

\newpage


\section{Chomsky-Hierarchie}

\vspace{0.5em}

\subsection{Typ 0 ($\mathcal{L}0$) - Phrasenstrukturgrammatiken}

\begin{itemize}
    \item Beliebige Kombination aus T- und NT-Symbolen
\end{itemize}

\vspace{0.5em}

\subsection{ Typ 1 ($\mathcal{L}1$) - Kontextsensitive Grammatiken}

\begin{itemize}
    \item $|l|\leq|r|$
    \item L"ange des Wortes steigt
    \item $S\rightarrow\varepsilon$ erlaubt, wenn S auf \textbf{keiner} rechten Seite einer Regel steht!
\end{itemize}

\raggedright
\textbf{Beispiel:}

$S \rightarrow S'$ | $\varepsilon$ \\
$S' \rightarrow aS'Bc$ | $abc$ \\
$cB \rightarrow Bc$ \\
$bB \rightarrow bb$ 

\vspace{0.5em}

Das Nichtterminal S' braucht man nur, damit die Bedingung der Sonderregel erfüllt ist. 
Das Nichtterminal B wird mal zur Satzform Bc und mal zu bb, je nachdem ob B im \textbf{Kontext} c oder b steht. 

\vspace{0.5em}

\subsection{Typ 2 ($\mathcal{L}2$) - Kontextfreie Grammatiken}

Beim Ableiten in Typ-1-Grammatiken muss man immer aufpassen, dass das Nichtterminal auch im richtigen Kontext steht. 
Das Erzeugen von Sätzen ist viel leichter, wenn die Grammatik kontextfrei ist. 

Eine Grammatik G ist vom Typ 2, wenn sie vom Typ 1 ist und zusätzlich auf der linken Seite jeder Regel genau \textbf{ein} Nichtterminal steht!

\begin{itemize}
    \item $l \in V$
    \item $X \rightarrow \varepsilon$ immer erlaubt
\end{itemize}

\vspace{0.5em}

\subsection{Typ 3 ($\mathcal{L}3$) - Reguläre Grammatik}

Eine Grammatik G ist vom Typ 3, wenn sie vom Typ 2 ist und zusätzlich folgende Regeln hat:

\begin{itemize}
    \item $A \rightarrow b$
    \item $A \rightarrow bC$
    \item $A \rightarrow \varepsilon$
\end{itemize}


\newpage

\section{Deterministischer Endlicher Automat (DEA)}

\hl{Eine DEA M ist ein 5-Tupel (Q, $\Sigma$, $\delta$, $q_0$, F):}

\begin{itemize}
    \item Q: endliche Zustandsmenge
    \item $\Sigma$: endliches Alphabet
    \item $\delta$: $Q \times \Sigma \rightarrow Q$ Übergangsfunktionen
    \item $q_0$: Startzustand
    \item F: Menge der akzeptierten Endzust"ande
\end{itemize}

\vspace{2em}
Beispiel:
\vspace{1em}

\includegraphics[width=0.4\textwidth]{DEA-00.png}

\begin{itemize}
    \item Q $= \{q_0, q_1, q_2\}$
    \item $\Sigma =$ \{0, 1\}
    \item $q_0 = q_0$
    \item F $= q_2$
    \item $\delta$:
    \begin{itemize}[label={}]
        \item $\delta(q_0, 0) = q_0$
        \item $\delta(q_0, 1) = q_1$
        \item $\delta(q_1, 0) = q_2$
        \item $\delta(q_1, 1) = q_1$
        \item $\delta(q_2, 0) = q_1$
        \item $\delta(q_2, 1) = q_1$
      \end{itemize}
\end{itemize}

\newpage

\section{Nicht-deterministischer Endlicher Automat (NEA)}

\hl{Eine NEA M ist ein 5-Tupel (Q, $\Sigma$, $\delta$, $q_0$, F):}

\begin{itemize}
    \item Q: endliche Zustandsmenge
    \item $\Sigma$: endliches Alphabet
    \item $\delta$: $Q \times \Sigma \rightarrow Q$ Übergangsfunktionen
    \item $q_0$: Menge der Startzust"ande
    \item F: Menge der akzeptierten Endzust"ande
\end{itemize}

\vspace{2em}
Beispiel:
\vspace{1em}

\begin{minipage}[h]{0.45\textwidth}
    $S \rightarrow aS$ | bS | cS | aA \\
    $A \rightarrow bB$ | cC \\
    $B \rightarrow aB$ | bB | cB | $\varepsilon$ \\
    $c \rightarrow aB$
\end{minipage}
\begin{minipage}[h]{0.45\textwidth}
    \includegraphics[width=0.7\textwidth]{NEA-00.png}
\end{minipage}

\newpage

\section{Äquivalenz von DEA und NEA}

\subsubsection{Satz von Rabin und Scott}

Jede von einem NEA akzeptierte Sprache L ist auch von einem DEA akzeptierbar.

\subsubsection{Potenzmengenkonstruktion}

!!!TODO!!!

\newpage

\section{Regex}

!!!TODO!!!

\subsubsection{Satz von Kleene}

Die Menge der durch regul"are Ausdr"ucke (Regex) beschreibbaren Sprachen ist genau die Menge der regul"aren Sprachen.

\vspace{0.5em}

$\rightarrow$ Alle endlichen Sprachen sind durch regul"are Ausdr"ucke beschreibbar

\newpage


\section{Pumping Lemma}

Das Pumping-Lemma wird verwendet, um zu beweisen, dass eine Sprache sicher nicht regul"ar ist.

!!!TODO!!!

\newpage

\section{Satz von Myhill und Nerode}

Eine Sprache L ist genau dann regul"ar, wenn der Index $R_L$ endlich ist!

\newpage


\section{Minimalautomaten}

!!!TODO!!!

\subsection{Table-Filling-Algorithmus}

\newpage


\section{Kontextfreie Sprachen ($\mathcal{L}2 $)}

\subsection{Chomsky Normalform (CNF)}

Regeln m"ussen folgende Formen haben:

\begin{itemize}
    \item $A \rightarrow BC$
    \item $A \rightarrow a$
    \item $S \rightarrow \varepsilon$
\end{itemize}

\subsection{Greibach Normalform}

Eine $\varepsilon$-freie, kontextfrei Grammatik mit folgenden Regeln:

\begin{itemize}
    \item $A \rightarrow aB_1B_2B_3...B_k$
    \item $k \geq 0$
\end{itemize}

\subsection{Konvertierung}

!!!TODO!!!

\newpage


\section{Kellerautomaten}

\hl{Ein Kellerautomat (PDA) M ist ein 6-Tupel (Q, $\Sigma$, $\Gamma$, $\delta$, $q_0$, $\#$):}

\begin{itemize}
    \item Q: endliche Zustandsmenge
    \item $\Sigma$: endliches Bandalphabet
    \item $\Gamma$: endliches Kelleralphabet
    \item $\delta$: "Ubergansfunktionen
    \item $q_0$: Startzustand ($q_0 \in Q$)
    \item $\#$: "Urspr"ungliches Kellersymbol ($q_0 \in \Gamma$)
\end{itemize}

%Es gibt keinen expliziten Endzustand. Das Ende ist erreicht, wenn auf dem Stack kein Element mehr liegt, oder das Wort komplett gelesen wurde.

\vspace{1em}

Akzeptanz:
\begin{itemize}
    \item Kein akzeptierender Endzustand!
    \item Akzeptanzkriterien für W"orter $x \in |Sigma^*$:
    \begin{enumerate}
        \item Wort komplett gelesen
        \item Keller (Stack leer)
    \end{enumerate}
\end{itemize}

\vspace{1em}

Nicht-Determinismus:
\begin{itemize}
    \item Mehrere simultane "Uberg"ange m"oglich
    \item Spontane "Uberg"ange ($a = \varepsilon$) m"oglich
\end{itemize}

\vspace{1em}

Konfiguration eines PDA gegeben durch 3-Tupel ($Q$, $\Sigma^*$, $\Gamma^*$):
\begin{itemize}
    \item $q \in Q$: Momentaner Zustand
    \item $w' \in \Sigma^*$: Noch zu lesender Anteil der Eingabe
    \item $\gamma \in \Gamma^*$: Aktueller Kellerinhalt
\end{itemize}

\vspace{1em}

"Ubergansfunktion:
\begin{itemize}
    \item $\delta$(q, a, A) $\ni$ ($q', B_1B_2...B_k$)
    \item Wenn Automat in Zustand q ist, das Symbol a liest und A oben auf Stack liegt, wechselt er in Zustand $q'$ und ersetzt das A auf dem Stack durch $B_1B_2...B_k$
\end{itemize}

\newpage



\section{CYK-Algorithmus}

\end{document}